%\documentclass[14pt]{beamer}
\documentclass[pdf,aspectratio=169]{beamer}
%\documentclass[pdf]{beamer}
%\mode<presentation>{}
%% preamble
\usepackage{svg}
\usetheme[]{Boadilla}
\usepackage{subfig}
\usepackage{graphicx}
\usepackage{ulem}
\usepackage{enumitem}
\usepackage{amsmath,amssymb,amsfonts}
\usepackage{tikz}
\usetikzlibrary{intersections,backgrounds}
\setitemize{label=\usebeamerfont*{itemize item}%
\usebeamercolor[fg]{itemize item}
\usebeamertemplate{itemize item}}
\usepackage{media9}
%\usepackage{pdfpcnotes}
\usepackage{hyperref}
\usepackage{xcolor}
\usepackage[labelformat=empty]{caption}
\setbeamertemplate{footline}[frame number]{}
\setbeamertemplate{footline}{}
\setbeamertemplate{page number in head/foot}{}
\setbeamertemplate{headline}{}
\setbeamertemplate{navigation symbols}{}
\newcommand{\backupbegin}{
   \newcounter{finalframe}
   \setcounter{finalframe}{\value{framenumber}}
}
\newcommand{\backupend}{
   \setcounter{framenumber}{\value{finalframe}}
}
\title{\textbf{Computational methods for nuclear engineering}}
\subtitle{Workshop on Computational Nuclear Science and Engineering}
\date{2021-07-12}
\institute{IAEA}
\titlegraphic{\includegraphics[width=\textwidth,height=.3\textheight,keepaspectratio]{img/HFIR_reload.jpg}}
\author[Touran]{Nick Touran, Ph.D., PE \\ \texttt{nick@whatisnuclear.com} }
\begin{document}
%% title frame

\begin{frame}
\titlepage
\end{frame}

\begin{frame}{Questions needing answers in reactor development}
        \begin{itemize}
            \item \textbf{Radiation transport} Where are the neutrons and photons? Where
                are they headed? How fast are nuclei splitting? What power are they
                generating, and where?
            \item \textbf{Thermal/hydraulics} How fast must we flow coolant to carry away
                the heat? What are the resulting pressures and vibrations?
            \item \textbf{Fuel performance} What are the mechanical and chemical
                dynamics of fuel system given the nuclear reactions and irradiation?
            \item \textbf{Mechanical} What are the mechanical loads amongst the fuel
                assemblies and/or other structures?
            \item \textbf{Plant systems} How do the pumps, pipes, heat exchangers,
                instrumentation, etc. perform in expected and postulated on- and off-normal
                conditions?
        \end{itemize}
\end{frame}

\begin{frame}{Answers can come from experiments}
\begin{figure}
\begin{tabular}{cccc}
\subfloat{\includegraphics[width=0.2\textwidth,height=0.4\textheight,keepaspectratio]{img/zppr.jpg}} &
\subfloat{\includegraphics[width=0.2\textwidth,height=0.4\textheight,keepaspectratio]{img/are_mockup.jpg}} &
\subfloat{\includegraphics[width=0.2\textwidth,height=0.4\textheight,keepaspectratio]{img/milling_yellow.jpg}} &
\subfloat{\includegraphics[width=0.2\textwidth,height=0.4\textheight,keepaspectratio]{img/appr1-core-in-crit-facility.jpg}} \\
\subfloat{\includegraphics[width=0.2\textwidth,height=0.4\textheight,keepaspectratio]{img/hengelo_sg_test_facility}} &
\subfloat{\includegraphics[width=0.2\textwidth,height=0.4\textheight,keepaspectratio]{img/superheat-critical.jpg}}&
\subfloat{\includegraphics[width=0.2\textwidth,height=0.4\textheight,keepaspectratio]{img/fsv_psrv_test.jpg}} &
\subfloat{\includegraphics[width=0.2\textwidth,height=0.4\textheight,keepaspectratio]{img/mtr_1955_hd.6d.017.jpg}}
\end{tabular}
\caption{Some historical nuclear experiments}
\end{figure}
\end{frame}

\begin{frame}{A Milling-Yellow solution shows fluid strain optically}
\begin{columns}
    \begin{column}{0.5\textwidth}
        \begin{figure}[ht]
        \centering
        \includegraphics[height=0.75\textheight,width=\textwidth,keepaspectratio]{img/milling_yellow}
            \caption{\small Flow patterns due to flow double refraction}
        \end{figure}
    \end{column}
    \begin{column}{0.5\textwidth}
        \begin{figure}[ht]
        \centering
        \includegraphics[height=0.75\textheight,width=\textwidth,keepaspectratio]{img/milling_yellow_rig}
            \caption{\small The rig used to capture the image}
        \end{figure}
    \end{column}
\end{columns}
    \centering \tiny HRE Progress Report July 1958
\end{frame}

\begin{frame}{Analog computer to measure the power coefficient of reactivity}
\begin{columns}
    \begin{column}{0.5\textwidth}
        \begin{figure}[ht]
        \centering
        \includegraphics[height=0.85\textheight,width=\textwidth,keepaspectratio]{img/hallam-analog-computer.png}
        \caption{\small Analog computer at Hallam}
        \end{figure}
    \end{column}
    \begin{column}{0.5\textwidth}
        \begin{figure}[ht]
        \centering
        \includegraphics[height=0.85\textheight,width=\textwidth,keepaspectratio]{img/hallam-kinetic-circuit-diagram}
        \caption{\small The schematics}
        \end{figure}
    \end{column}
\end{columns}
    \centering \tiny \url{https://babel.hathitrust.org/cgi/pt?id=mdp.39015095040682&view=1up&seq=32}
\end{frame}

\begin{frame}{History of Computational Nuclear Engineering}
        \begin{figure}[ht]
        \centering
            \includegraphics[height=0.90\textheight,width=\textwidth,keepaspectratio]{img/ebr2-univac.png}
    \caption{\small Digital computers were used in 1955 for reactor design
            (\url{https://babel.hathitrust.org/cgi/pt?id=mdp.39015003993188&view=1up&seq=1})}
        \end{figure}
\end{frame}

%\begin{frame}{About me: Big moments}
%\begin{columns}
%    \begin{column}{0.7\textwidth}
%        \begin{itemize}
%            \item Got interested in something I consider important
%            \item Happened upon a fringe outlet for that interest
%            \item Eventually got serious about studies
%            \item Chose extra-fringe/risky skillset at end of undergrad
%            \item Reached out relatively hard to cool-sounding internships
%            \item Luck: Had skills the Bill Gates firm needed
%            \item Delivered cool stuff to the project
%        \end{itemize}
%    \end{column}
%    \begin{column}{0.3\textwidth}
%        \begin{figure}[ht]
%        \centering
%        \includegraphics[height=0.80\textheight,width=0.9\textwidth,keepaspectratio]{img/teton-cliff.jpg}
%            \caption{\tiny Cliff in the Tetons from my INL internship}
%        \end{figure}
%    \end{column}
%\end{columns}
%\end{frame}



\begin{frame}{Today, computation is used for a variety of aims}
\begin{columns}
    \begin{column}{0.7\textwidth}
        \begin{itemize}
            \item General engineering/design without the challenges of classical analysis
            \item Explore a new space in the reactor design space
            \item Seek insight into some complexity with ultra-high fidelity
            \item Run many perturbed cases to seek a multi-objective optimum
            \begin{itemize}
                \item in trade studies backing design decisions
                \item in preparation for targeted experiments 
                \item to estimate complex system-level sensitivities
            \end{itemize}
        \end{itemize}
    \end{column}
    \begin{column}{0.3\textwidth}
        \begin{figure}[ht]
        \centering
        \includegraphics[height=0.80\textheight,width=0.9\textwidth,keepaspectratio]{img/teton-cliff.jpg}
            \caption{\tiny Cliff in the Tetons}
        \end{figure}
    \end{column}
\end{columns}
\end{frame}

\begin{frame}{Basic approach to deterministic computational nuclear engineering}
    \begin{itemize}
        \item Write a balance statement
        \item Make assumptions to simplify as appropriate
        \item Specify boundary conditions
        \item Discretize the solution space considering geometry and material properties
        \item Cast balance statement into matrix form
        \item Solve matrix equation numerically
    \end{itemize}

\end{frame}

\section{Neutronics}
\begin{frame}{The Neutron Transport Equation is a balance statement}
    \begin{center}
        \begin{figure}[ht]
        \centering
        \includegraphics[height=0.80\textheight,width=0.9\textwidth,keepaspectratio]{img/neutron-transport-eqn}
            \caption{\tiny The neutron transport equation}
        \end{figure}
    \end{center}
    Given a solution to this, you get nuclear reaction rates, which give:
    \begin{itemize}
        \item Gamma source distribution (leads to another transport solve)
        \item Power distribution (cooling, distortions)
        \item Dose rate distribution (material damage, detector signal)
        \item Dynamics (operations, safety)
    \end{itemize}

\end{frame}

\begin{frame}{The domain: a nuclear core}
\begin{columns}
    \begin{column}{0.5\textwidth}
        \begin{figure}[ht]
        \centering
        \includegraphics[height=0.80\textheight,width=\textwidth,keepaspectratio]{img/hallam_core}
            \caption{\small The Hallam core}
        \end{figure}
    \end{column}
    \begin{column}{0.5\textwidth}
        \begin{figure}[ht]
        \centering
        \includegraphics[height=0.85\textheight,width=\textwidth,keepaspectratio]{img/hallam_cell}
        \caption{\small A large 'unit cell'}
        \end{figure}
    \end{column}
\end{columns}
\end{frame}

\begin{frame}{Spatial discretization}
\begin{columns}
    \begin{column}{0.5\textwidth}
        \begin{figure}[ht]
        \centering
        \includegraphics[height=0.85\textheight,width=\textwidth,keepaspectratio]{img/dif3d-triz.png}
        \caption{\small A Triangular-Z volume element}
        \end{figure}
    \end{column}
    \begin{column}{0.5\textwidth}
        \begin{figure}[ht]
        \centering
        \includegraphics[height=0.85\textheight,width=\textwidth,keepaspectratio]{img/dif3d-trimesh.png}
        \caption{\small A 120° domain}
        \end{figure}
    \end{column}
\end{columns}
    \centering \tiny From DIF3D: A Code to Solve One-, Two-, and Three-Dimensional
    Finite-Difference Diffusion Theory Problems
\end{frame}

\begin{frame}{Some approximations to the transport equation}

Steady state, multi-group diffusion:
\begin{equation}
\left(D_{g}+\Sigma_{g}\right)\phi_{g}-\sum_{g^{\prime}\ne g}T_{gg^{\prime}}\phi_{g}=\frac{1}{\lambda}\chi_{g}\sum_{g^{\prime}=1}^{G}F_{g^{\prime}}\phi_{g^{\prime}},\label{eq:group-balance-orig}
\end{equation}


\begin{equation}
M=\left[\begin{array}{cccc}
\left[A_{1}\right] &  &  & \left[0\right]\\
 & \left[A_{2}\right]\\
 &  & \ddots\\
\left[0\right] &  &  & \left[A_{G}\right]
\end{array}\right]-\left[\begin{array}{ccccc}
[0] & \left[T_{12}\right] & \cdots & \cdots & \left[T_{1G}\right]\\
\left[T_{21}\right] & [0] &  &  & \vdots\\
\vdots &  & \ddots &  & \vdots\\
\vdots &  &  & \ddots & \left[T_{G-1,G}\right]\\
\left[T_{G1}\right] & \left[T_{G2}\right] & \cdots & \left[T_{G,G-1}\right] & [0]
\end{array}\right],
\end{equation}

Eigenvalue equation:
\begin{equation}
M\phi=\frac{1}{\lambda}B\phi,\label{eq:generalizedEigenproblem}
\end{equation}

\end{frame}


\begin{frame}[t]{Building the matrices}
    \begin{columns}[onlytextwidth, t]
    \column{0.5\textwidth}
    \begin{equation}
    \resizebox{0.8 \textwidth}{!}
    {
        $$
        A_{g}=\left[\begin{array}{ccccccc}
        \ddots & \ddots &  & \ddots\\
         & \left[A_{gx}^{J-1,k}\right] & \left[A_{gy}^{J,k}\right] &  & \left[A_{gz}^{J-1,k+1}\right]\\
         &  & \left[A_{gx}^{J,k}\right] & \left[0\right] &  & \left[A_{gz}^{J,k+1}\right]\\
         & \text{} &  & \left[A_{gx}^{1,k+1}\right] & \left[A_{gy}^{2,k+1}\right] &  & \ddots\\
         &  &  &  & \ddots & \ddots
        \end{array}\right],
        $$
    }
    \end{equation}

    \begin{equation}
    \resizebox{0.8 \textwidth}{!}
    {
        $$
        \left[A_{gx}^{j,k}\right]=\left[\begin{array}{ccccccc}
        b_{1} & -a_{2}^{x}\\
        -a_{2}^{x} & b_{2} & -a_{3}^{x}\\
         & \ddots & \ddots & \ddots\\
         &  & \ddots & \ddots & \ddots\\
         &  &  & \ddots & \ddots & \ddots\\
         &  &  &  & -a_{I-1}^{x} & b_{I-1} & -a_{I}^{x}\\
         &  &  &  &  & -a_{I}^{x} & b_{I}
        \end{array}\right]_{ikg},
        $$
    }
    \end{equation}
        \column{0.5\textwidth}
            \centering
            \begin{figure}[ht]
                \includegraphics[height=0.65\textheight,width=\textwidth,keepaspectratio]{img/fd_matrix.jpg}
            \caption{\small A 7-banded diffusion matrix for one group in 1/3-core 3D triangular geometry}
            \end{figure}
    \end{columns}
\end{frame}

\begin{frame}{We have high fidelity deterministic codes these days}
\begin{columns}
    \begin{column}{0.5\textwidth}
        \begin{itemize}
            \item Unstructured mesh
            \item Detailed treatment of scattering
            \item Requires cross section averaging lattice calc
            \item Powerful matrix solvers (e.g. LAPACK, PETSc, Trilinos)
        \end{itemize}
    \end{column}
    \begin{column}{0.5\textwidth}
        \begin{figure}[ht]
        \centering
            \includegraphics[height=0.55\textheight,width=\textwidth,keepaspectratio]{img/rattlesnake-atr-mesh}
        \caption{\small A fine mesh of the ATR from
            https://rattlesnake.inl.gov/SitePages/Home.aspx}
        \end{figure}
    \end{column}
\end{columns}
\end{frame}

\begin{frame}{Nuclear data evaluations undergo significant computation}
        \begin{itemize}
            \item Many measurements are made
            \item Nuclear models run to interpolate between data
            \item Complex statistics and Bayesian inference to get 'best' fit
        \end{itemize}
\begin{columns}
    \begin{column}{0.5\textwidth}
        \begin{figure}[ht]
        \centering
        \includegraphics[height=0.55\textheight,width=\textwidth,keepaspectratio]{img/pu239-endf8}
        \caption{\small The evaluated data}
        \end{figure}
    \end{column}
    \begin{column}{0.5\textwidth}
        \begin{figure}[ht]
        \centering
            \includegraphics[height=0.55\textheight,width=\textwidth,keepaspectratio]{img/pu239-exfor}
        \caption{\small A small selection of experimental data}
        \end{figure}
    \end{column}
\end{columns}
    \centering \tiny \url{https://www.oecd-nea.org/janisweb}
\end{frame}

\begin{frame}{Monte Carlo: explicitly track particle histories}
\begin{columns}
    \begin{column}{0.5\textwidth}
        \begin{itemize}
            \item No geometry approximations
            \item No angular approximations
            \item No multi-group approximations
            \item No cross section averaging
            \item \textbf{But...} lots and lots of histories needed
            \item Big computers
            \item Lots of time
        \end{itemize}
    \end{column}
    \begin{column}{0.5\textwidth}
        \begin{figure}[ht]
        \centering
            \includegraphics[height=0.75\textheight,width=\textwidth,keepaspectratio]{img/monte_carlo_casino}
            \caption{\small The Monte Carlo Casino (CC-BY-2.0 Lolipouette)} 
        \end{figure}
    \end{column}
\end{columns}
\end{frame}

\begin{frame}{Monte Carlo: basics}
        \begin{itemize}
            \item Roll dice: sample distance to next interaction
            \item Sample from cross section data to choose reaction
            \item Sample from cross section data to choose outgoing energy and direction
            \item Repeat billions of times, tallying up track length \& reaction rates
        \end{itemize}
\begin{columns}
    \begin{column}{0.5\textwidth}
        \begin{figure}[ht]
        \centering
            \includegraphics[height=0.55\textheight,width=\textwidth,keepaspectratio]{img/lwr-xs-sq-line}
            \caption{\tiny Look at the macroscopic cross sections at the particle's current energy} 
        \end{figure}
    \end{column}
    \begin{column}{0.5\textwidth}
        \begin{figure}[ht]
        \centering
            \includegraphics[height=0.55\textheight,width=\textwidth,keepaspectratio]{img/mc-reaction-choice}
            \caption{\tiny A discrete CDF for material choice} 
        \end{figure}
    \end{column}
\end{columns}
\end{frame}

\begin{frame}{Transmutation and decay}

\begin{equation}
\frac{\partial N_{i}}{\partial t}=-N_{i}\left(\sum_{g=1}^{G}\sigma_{a,g}^{i}\phi_{g}+\lambda_{i}\right)+\sum_{j=1}^{I}N_{j}\left(\sum_{g=1}^{G}\left(\gamma_{g}^{j\rightarrow i}\phi_{g}\right)+\lambda_{j\rightarrow i}\right),\label{eq:transmutation_rates}
\end{equation}

    \begin{center}
    \includegraphics[height=0.6\textheight,width=\textwidth,keepaspectratio]{img/transmutation}
    \end{center}
\end{frame}


\section{Fluid flow and heat transfer}
\begin{frame}{Fluid flow and heat transfer}
\begin{columns}
    \begin{column}{0.5\textwidth}
        \begin{itemize}
            \item Essential in core; highly coupled 
            \item Essential in entire plant: pipes, pumps, and heat exchangers
            \item Fundamental to safety performance
            \item Often dependent on empirical correlations
            \item 
        \end{itemize}
    \end{column}
    \begin{column}{0.5\textwidth}
        Governing equations: Navier-Stokes
        \begin{itemize}
            \item Conservation of mass 
            \item Conservation of energy
            \item Conservation of momentum
        \end{itemize}
        Approach is familiar: discretize spatially, cast to matrix looking up material
        properties, and solve

    \end{column}
    \end{columns}
\end{frame}


\begin{frame}{Important couplings between thermal/hydraulics and neutronics}
        \begin{itemize}
            \item Macroscopic nuclear cross sections are temperature dependent (thermal
                expansion, soluble boron)
            \item Microscopic nuclear cross sections are temperature dependent (Doppler effect)
            \item Flow-induced pressure can distort the core, causing changes in
                macroscopic cross section
            \item In fluid fuel reactor or severe accidents, flow moves the fuel and
                delayed neutron precursors
        \end{itemize}
\end{frame}

\begin{frame}{Fundamental T/H codes}
\begin{columns}
    \begin{column}{0.7\textwidth}
        \begin{itemize}
            \item Single channel
                \begin{itemize}
                    \item Typically 1-D fluid flow
                    \item Hot channel factors and correlations for peaking, turbulent mixing,
                        heat transfer, friction factor, and phase slip
                    \item One phase (LMFBR or GCR) or two phase (PWR and BWR)
                    \item 1- or 2-D conduction models into fuel and structures
                    \item Full-core transient calculations trivial
                \end{itemize}
            \item Subchannel 
                \begin{itemize}
                    \item Communicating subchannels (mixing, conduction)
                    \item Intra-assembly heat transfer, often 3-D conduction to fuel
                    \item Less reliant on hot channel factors, but still needs
                        correlations
                    \item Full-core static calculations routine, transients doable
                \end{itemize}
            \end{itemize}
    \end{column}
    \begin{column}{0.3\textwidth}
        \begin{figure}[ht]
        \centering
            \includegraphics[height=0.55\textheight,width=\textwidth,keepaspectratio]{img/cortran-subchannels}
            \caption{\tiny Subchannels from CORTRAN \url{https://www.osti.gov/servlets/purl/6881496}}
        \end{figure}
    \end{column}
\end{columns}
\end{frame}

\begin{frame}{Systems codes}
\begin{columns}
    \begin{column}{0.7\textwidth}
    \begin{itemize}
        \item Simulate the entire plant in various operational and off-normal conditions
        \item Front and center in licensing
        \item Includes representations of major equipment in balance of plant
            (sometimes in lower fidelity than the core)
        \item Coupled to nuclear kinetics codes
        \item Often incorporate subchannel code for core
    \end{itemize}
    \end{column}
    \begin{column}{0.3\textwidth}
        \begin{figure}[ht]
        \centering
            \includegraphics[height=0.55\textheight,width=\textwidth,keepaspectratio]{img/th-discretization}
            \caption{\tiny Discretization in TOPAZ
            \url{https://www.osti.gov/servlets/purl/5437631}}
        \end{figure}
    \end{column}
\end{columns}
\end{frame}

\begin{frame}{Computational Fluid Dynamics}

\begin{columns}
    \begin{column}{0.7\textwidth}
    \begin{itemize}
        \item High-resolution mesh 
        \item Several methods with different turbulence models
        \item Multi-phase flow still under development
        \item Used to verify use of correlations in faster subchannel codes
        \item Used for thermal striping, stratification, etc.
        \item Used during trade studies
        \item Useful in exploring and understanding new design space lacking in
            experiments
        \item Helps design the most impactful experiments
    \end{itemize}
    \end{column}
    \begin{column}{0.3\textwidth}
        \begin{figure}[ht]
        \centering
            \includegraphics[height=0.55\textheight,width=\textwidth,keepaspectratio]{img/plenum-mixing}
            \caption{\tiny Obabko, 2015
            \url{https://www.mcs.anl.gov/papers/P5386-0715.pdf}}
        \end{figure}
    \end{column}
\end{columns}
\end{frame}

\begin{frame}{Trivial T/H solution for 1-D single-phase flow}
\begin{columns}
    \begin{column}{0.5\textwidth}
        \begin{figure}[ht]
        \centering
            \includegraphics[height=0.55\textheight,width=\textwidth,keepaspectratio]{img/th-code-massflow}
            \caption{\tiny Compute the required mass flow rate given the assembly power } 
        \end{figure}
    \end{column}
    \begin{column}{0.5\textwidth}
        \begin{figure}[ht]
        \centering
            \includegraphics[height=0.55\textheight,width=\textwidth,keepaspectratio]{img/th-code-1d}
            \caption{\tiny Given mass flow rate, compute the axial coolant temperature
            distribution} 
        \end{figure}
    \end{column}
\end{columns}
\end{frame}

\section{Fuel Performance}

\begin{columns}
    \begin{column}{0.5\textwidth}
        \begin{figure}[ht]
        \centering
            \includegraphics[height=0.55\textheight,width=\textwidth,keepaspectratio]{img/th-code-massflow}
            \caption{\tiny Compute the required mass flow rate given the assembly power } 
        \end{figure}
    \end{column}
    \begin{column}{0.5\textwidth}
        \begin{figure}[ht]
        \centering
            \includegraphics[height=0.55\textheight,width=\textwidth,keepaspectratio]{img/th-code-1d}
            \caption{\tiny Given mass flow rate, compute the axial coolant temperature
            distribution} 
        \end{figure}
    \end{column}
\end{columns}
\end{frame}


\begin{frame}[plain,c]
	\begin{center}
	\Huge Discussion!\\
    \includegraphics[width=0.1\textwidth,keepaspectratio]{img/reactor.png}\\
    \normalsize More info at whatisnuclear.com \\
    \normalsize Or email me: nick@whatisnuclear.com
	\end{center}
\end{frame}

\appendix
\backupbegin




\backupend

\end{document}
